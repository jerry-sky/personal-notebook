%%%%%%%%%%%%%%%%%
% This is an example CV created using altacv.cls (v1.3, 10 May 2020) written by
% LianTze Lim (liantze@gmail.com), based on the
% Cv created by BusinessInsider at http://www.businessinsider.my/a-sample-resume-for-marissa-mayer-2016-7/?r=US&IR=T
%
%% It may be distributed and/or modified under the
%% conditions of the LaTeX Project Public License, either version 1.3
%% of this license or (at your option) any later version.
%% The latest version of this license is in
%%    http://www.latex-project.org/lppl.txt
%% and version 1.3 or later is part of all distributions of LaTeX
%% version 2003/12/01 or later.
%%%%%%%%%%%%%%%%

%% If you are using \orcid or academicons
%% icons, make sure you have the academicons
%% option here, and compile with XeLaTeX
%% or LuaLaTeX.
% \documentclass[10pt,a4paper,academicons]{altacv}

%% Use the "normalphoto" option if you want a normal photo instead of cropped to a circle
% \documentclass[10pt,a4paper,normalphoto]{altacv}

\documentclass[10pt,a4paper,ragged2e,withhyper]{altacv}

%% AltaCV uses the fontawesome5 and academicon fonts
%% and packages.
%% See http://texdoc.net/pkg/fontawesome5 and http://texdoc.net/pkg/academicons for full list of symbols. You MUST compile with XeLaTeX or LuaLaTeX if you want to use academicons.

% Change the page layout if you need to
\geometry{left=1cm,right=1cm,top=0.7cm,bottom=0.3cm,columnsep=1cm}

% The paracol package lets you typeset columns of text in parallel
\usepackage{paracol}


% Change the font if you want to, depending on whether
% you're using pdflatex or xelatex/lualatex
\ifxetexorluatex
  % If using xelatex or lualatex:
  \setmainfont{Lato}
\else
  % If using pdflatex:
  \usepackage[T1]{fontenc}

  \usepackage{merriweather}
%   \usepackage[sfdefault, default]{FiraSans}
%   \renewcommand*\oldstylenums[1]{{\firaoldstyle #1}}

  \usepackage{FiraMono}
\fi

% Change the colours if you want to
\definecolor{VividPurple}{HTML}{3E0097}
\definecolor{SlateGrey}{HTML}{2E2E2E}
\definecolor{LightGrey}{HTML}{666666}
% \colorlet{name}{black}
\colorlet{tagline}{VividPurple}
\colorlet{heading}{VividPurple}
\colorlet{headingrule}{VividPurple}
% \colorlet{subheading}{PastelRed}
\colorlet{accent}{VividPurple}
\colorlet{emphasis}{SlateGrey}
\colorlet{body}{LightGrey}

% Change some fonts, if necessary
% \renewcommand{\namefont}{\Huge\rmfamily\bfseries}
% \renewcommand{\personalinfofont}{\footnotesize}
% \renewcommand{\cvsectionfont}{\LARGE\rmfamily\bfseries}
% \renewcommand{\cvsubsectionfont}{\large\bfseries}

% Change the bullets for itemize and rating marker
% for \cvskill if you want to
\renewcommand{\itemmarker}{{\small\textbullet}}
\renewcommand{\ratingmarker}{\faCircle}

\photoL{2.5cm}{picture}
\begin{document}
\name{Jerzy Wroczyński}
\tagline{Computer science student}
\personalinfo{%
  % Not all of these are required!
  % You can add your own with \printinfo{symbol}{detail}
  \email{jwroczynski@gmail.com}
  \homepage{jerry-sky.me}
  \github{jerry-sky}
%   \linkedin{jerzy-wroczy\%C5\%84ski-bb0333205}
  \printinfo{\faLinkedin}{LinkedIn}[https://www.linkedin.com/in/jerzy-wroczy\%C5\%84ski-bb0333205/]
  \printinfo{\faMapMarker}{Wrocław, Poland}[https://goo.gl/maps/trCpuWtcrPC6neNq9]
  %% You can add your own arbitrary detail with
  %% \printinfo{symbol}{detail}[optional hyperlink prefix]
  % \printinfo{\faPaw}{Hey ho!}
  %% Or you can declare your own field with
  %% \NewInfoFiled{fieldname}{symbol}[optional hyperlink prefix] and use it:
  % \NewInfoField{gitlab}{\faGitlab}[https://gitlab.com/]
  % \gitlab{your_id}
}

\makecvheader

\smallskip
% \textit{<short job-specific motivational clause>}

%% Depending on your tastes, you may want to make fonts of itemize environments slightly smaller
\AtBeginEnvironment{itemize}{\small}

%% Set the left/right column width ratio to 6:4.
\columnratio{0.6}

% Start a 2-column paracol. Both the left and right columns will automatically
% break across pages if things get too long.
\begin{paracol}{2}

\cvsection{Education}

\cvachievement{\href{https://pwr.edu.pl/en}{Wrocław University of Science and Technology}}{Computer Science (B.Eng.) 2018–Present}

\divider

\cvachievement{\href{https://liceum.pwr.edu.pl/}{Academic High School of Wrocław University of Science and Technology (ALO PWr)}}{Mathematics, Physics, IT; attended 2015–2018}

\divider

\cvachievement{\href{https://wca.wroc.pl/studencki-program-stypendialny-stypendium-dla-studentow-bedacych-laureatami-ogolnopolskich-olimpiad-przedmiotowych-i-konkursow-lista-stypendystow}{Scholarship}}{Received scholarship from the City of Wrocław for the first academic year (2018–2019)}

\divider

\cvachievement{“Studium Talent”}{Completed a study program (with grade 5.0 on scale from 2.0 to 5.5) for High School students (organized by WUST) that granted a place in the University (skipping the normal university admission process); attended the program 2017–2018}


\cvsection{Projects and Experiences}

\cvevent{E-commerce website}{Angular (2+), ExpressJS (with TypeScript), MySQL, NodeJS with NPM, Linux, Bash}{2019–2020}{}

\small Krzemień Wroczyński (previous version) — simple e-commerce website I have built for my father’s business
\smallskip

\begin{itemize}
\item Used Angular (2+) for the front-end SPAs (for both the admin panel and the store website).
\item Implemented an API in Express.js that connects to a MySQL database that stores information about available products and pending orders.
\item Created (thanks to TypeScript) a strict model to greatly improve type-safety across the whole system.
\item Set up a Linux server using Bash and various NPM packages/ programs.
\end{itemize}

\divider

\cvevent{Repository notebooks, \href{https://github.com/marketplace/actions/vyrow}{\textit{VYROW}}}{Git, Markdown, \LaTeX, Python, Bash, Pandoc, GitHub Actions}{2020–Present}{}
\begin{itemize}
\item Created special repositories for collecting programs, scripts and notes
\item Actively using my \href{https://personal.jerry-sky.me}{Personal Notebook repository} as a way to organize my config files, scripts and, perhaps most importantly, notes on various subjects.
\item Actively using (during academic year) my \href{https://academic.jerry-sky.me}{Academic Notebook repository} for collecting all of my programs and notes that I have written during my studies.
\item Created a GitHub Action that automatically renders all Markdown documents into HTML documents; (\href{https://github.com/marketplace/actions/vyrow}{VYROW — View Your Repository On the Web})
\end{itemize}

\divider

%% Switch to the right column. This will now automatically move to the second
%% page if the content is too long.
\switchcolumn

%% Switch to the right column. This will now automatically move to the second
%% page if the content is too long.
% \switchcolumn

\cvsection{Familiar with}

\begin{itemize}
\item Linux, Bash
\item Web dev: TypeScript, JavaScript, ExpressJS, Angular (2+), NodeJS, NPM
\item MySQL
\item Python
\item Java
\item C/C++
\item Git
\item Markdown, LaTeX, Pandoc
\item Blender 3D
\end{itemize}

\cvsection{Interests}

\begin{itemize}
  \item Solutions for note-taking and organization
  \item Unix-like OSes
  \item Web development
  \item Language learning
  \item Blender 3D — designing everyday objects and creating still scene images
\end{itemize}

\cvsection{Languages}

\cvskill{English \textit{(C1 level)}}{4}
% \divider

\cvskill{German \textit{(beginner B1)}}{2}
% \divider

\cvskill{Polish \textit{(native)}}{5}
% \divider

\bigskip

{\color{headingrule}\rule{\linewidth}{2pt}\par}

\bigskip

\cvevent{Creating 3D assets for a video game}{Blender 3D}{2019–2020}{}
\small\href{https://github.com/Qarian/Restaurant-rush}{Restaurant Rush (unreleased)}
\smallskip

\begin{itemize}
    \item Used Blender to create 3D assets for said game.
\end{itemize}

\divider

\cvevent{Web development (small projects)}{HTML, CSS, JavaScript, PHP, AngularJS, }{2014–2017}{}
\begin{itemize}
    \item Remote Timer — a small program that controls a timer on another computer
    \item A web application that I created for my classmates and myself; its purpose was to keep us informed about homework etc.
    \item Some other projects that haven’t been archived
\end{itemize}

\divider

\end{paracol}

\vspace{8pt}

\small\textit{I hereby give consent for my personal data included in the application to be processed for the purposes of the recruitment process in accordance with Art. 6 paragraph 1 letter a of the Regulation of the European Parliament and of the Council (EU) 2016/679 of 27 April 2016 on the protection of natural persons with regard to the processing of personal data and on the free movement of such data, and repealing Directive 95/46/EC (General Data Protection Regulation).}

\end{document}
