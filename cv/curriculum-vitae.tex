%%%%%%%%%%%%%%%%%
% This is an example CV created using altacv.cls (v1.3, 10 May 2020) written by
% LianTze Lim (liantze@gmail.com), based on the
% Cv created by BusinessInsider at http://www.businessinsider.my/a-sample-resume-for-marissa-mayer-2016-7/?r=US&IR=T
%
%% It may be distributed and/or modified under the
%% conditions of the LaTeX Project Public License, either version 1.3
%% of this license or (at your option) any later version.
%% The latest version of this license is in
%%    http://www.latex-project.org/lppl.txt
%% and version 1.3 or later is part of all distributions of LaTeX
%% version 2003/12/01 or later.
%%%%%%%%%%%%%%%%

%% If you are using \orcid or academicons
%% icons, make sure you have the academicons
%% option here, and compile with XeLaTeX
%% or LuaLaTeX.
% \documentclass[10pt,a4paper,academicons]{altacv}

%% Use the "normalphoto" option if you want a normal photo instead of cropped to a circle
% \documentclass[10pt,a4paper,normalphoto]{altacv}

\documentclass[10pt,a4paper,ragged2e,withhyper]{altacv}

%% AltaCV uses the fontawesome5 and academicon fonts
%% and packages.
%% See http://texdoc.net/pkg/fontawesome5 and http://texdoc.net/pkg/academicons for full list of symbols. You MUST compile with XeLaTeX or LuaLaTeX if you want to use academicons.

% Change the page layout if you need to
\geometry{left=1.25cm,right=1.25cm,top=1.5cm,bottom=1.5cm,columnsep=1.2cm}

% The paracol package lets you typeset columns of text in parallel
\usepackage{paracol}


% Change the font if you want to, depending on whether
% you're using pdflatex or xelatex/lualatex
\ifxetexorluatex
  % If using xelatex or lualatex:
  \setmainfont{Lato}
\else
  % If using pdflatex:
  \usepackage[default]{lato}
\fi

% Change the colours if you want to
\definecolor{VividPurple}{HTML}{3E0097}
\definecolor{SlateGrey}{HTML}{2E2E2E}
\definecolor{LightGrey}{HTML}{666666}
% \colorlet{name}{black}
\colorlet{tagline}{VividPurple}
\colorlet{heading}{VividPurple}
\colorlet{headingrule}{VividPurple}
% \colorlet{subheading}{PastelRed}
\colorlet{accent}{VividPurple}
\colorlet{emphasis}{SlateGrey}
\colorlet{body}{LightGrey}

% Change some fonts, if necessary
% \renewcommand{\namefont}{\Huge\rmfamily\bfseries}
% \renewcommand{\personalinfofont}{\footnotesize}
% \renewcommand{\cvsectionfont}{\LARGE\rmfamily\bfseries}
% \renewcommand{\cvsubsectionfont}{\large\bfseries}

% Change the bullets for itemize and rating marker
% for \cvskill if you want to
\renewcommand{\itemmarker}{{\small\textbullet}}
\renewcommand{\ratingmarker}{\faCircle}

\begin{document}
\name{Jerzy Wroczyński}
\tagline{Computer Science Student}
\personalinfo{%
  % Not all of these are required!
  % You can add your own with \printinfo{symbol}{detail}
  \email{jwroczynski@gmail.com}
  \homepage{jerry-sky.github.io}
  \github{jerry-sky}
  \location{Wrocław, Poland}
  %% You can add your own arbitrary detail with
  %% \printinfo{symbol}{detail}[optional hyperlink prefix]
  % \printinfo{\faPaw}{Hey ho!}
  %% Or you can declare your own field with
  %% \NewInfoFiled{fieldname}{symbol}[optional hyperlink prefix] and use it:
  % \NewInfoField{gitlab}{\faGitlab}[https://gitlab.com/]
  % \gitlab{your_id}
}

\makecvheader

%% Depending on your tastes, you may want to make fonts of itemize environments slightly smaller
\AtBeginEnvironment{itemize}{\small}

%% Set the left/right column width ratio to 6:4.
\columnratio{0.6}

% Start a 2-column paracol. Both the left and right columns will automatically
% break across pages if things get too long.
\begin{paracol}{2}

\cvsection{Education}

\cvachievement{\href{https://pwr.edu.pl/en}{Wrocław University of Science and Technology}}{Computer Science (B.Eng.) 2018–Present}

\divider

\cvachievement{\href{https://liceum.pwr.edu.pl/}{Academic High School of Wrocław University of Science and Technology}}{Mathematics, Physics, IT; attended 2015-2018}

\divider

\cvachievement{Scholarship}{Received scholarship from the City of Wrocław (the first academic year)}

\divider

\cvachievement{Studium Talent}{Completed a study program (with grade 5.0 on scale from 2.0 to 5.5) for High School students (organised by WUST) that granted a place in the University (skipping the normal university admission process)}


\cvsection{Projects and Experiences}

\cvevent{Repository notebooks}{Git, Markdown, Python, \texttt{bash}}{}{}
\begin{itemize}
\item Created special repositories for collecting programs, scripts and notes
\item Still actively using my \href{https://github.com/jerry-sky/personal-notebook#readme}{Personal Notebook repository} as a way to organise config files, scripts and, perhaps most importantly, notes on various subjects.
\item Created \href{https://github.com/jerry-sky/academic-notebook#readme}{Academic Notebook repository} for collecting all of my programs and notes that I have written during my studies.
\end{itemize}

\divider

\cvevent{E-commerce website}{Angular, Express.js (with TypeScript), MySQL}{}{}

\small\href{https://wroczynski.pl}{Krzemień Wroczyński} — simple e-commerce website I have built for my father’s business \textit{(currently working on much improved second version).}
\smallskip

\begin{itemize}
\item Used Angular (2+) for the front-end SPAs (for both the admin panel and the store website).
\item Implemented an API in Express.js that connects to a MySQL database that stores information about available products and pending orders.
\item Created (thanks to TypeScript) a strict model to greatly improve type-safety across the whole system.
\end{itemize}

\divider

\cvevent{Creating 3D assets for a video game}{Blender}{}{}
\small\href{https://github.com/Qarian/Restaurant-rush}{Restaurant Rush (unreleased)}
\smallskip

\begin{itemize}
    \item Used Blender to create 3D assets for the game.
\end{itemize}

\divider

\cvevent{General usage of Angular (JS and 2+)}{Angular}{}{}
\begin{itemize}
    \item e.g. \href{https://jerry-sky.github.io}{my Github Pages website}
    \item other projects that have not been archived
\end{itemize}

%% Switch to the right column. This will now automatically move to the second
%% page if the content is too long.
\switchcolumn

\cvsection{Familiar with}

\begin{itemize}
\item Angular (2+)
\item Express.js
\item TypeScript
\item JavaScript
\item Python
\item MySQL
\item C/C++
\item Git
\item Blender
\item Linux, \texttt{bash}
\item Markdown
\end{itemize}

\cvsection{Languages}

\cvskill{English \textit{(C1 level)}}{4}
% \divider

\cvskill{German \textit{(beginner B1)}}{2}
% \divider

\cvskill{Polish \textit{(native)}}{5}
% \divider

\end{paracol}

\end{document}
